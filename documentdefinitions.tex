\startenvironment documentdefinitions

\writestatus{environment}{Loading shared document definitions}
\writestatus{}           {version 2013.02.19 by Tim Steenvoorden}

% -----------------------------------------------------------------------------
% Catcodes
% -----------------------------------------------------------------------------

\define[1]\makecharacterother
  {\catcode`#1\othercatcode}

\makecharacterother |
%\makecharacteractive @ %@
%\def@#1@{\type{#1}}
%\def|#1|{\type{#1}}

% -----------------------------------------------------------------------------
% Language
% -----------------------------------------------------------------------------

\setuplanguage[en]
  [spacing=packed]
\setuplanguage[nl]
  [leftquote=\upperleftsinglesixquote,
   rightquote=\upperrightsingleninequote,
   leftquotation=\upperleftdoublesixquote,
   rightquotation=\upperrightdoubleninequote]

\setuplabeltext[nl]%TODO: to definitions?
  [chapter=Hoofdstuk~,
   appendix=Appendix~,
   seealso=zie~ook~]

\setupheadtext[nl]
  [content={Inhoudsopgave},
   tables={Lijst van tabellen},
   figures={Lijst van figuren},
   graphics={Lijst van grafieken},
   intermezzi={Lijst van intermezzo's},
   abbreviations={Lijst van afkortingen},
   logos={Lijst van logo's},
   units={Lijst van eenheden}]
   %pubs={Literatuur}
   %index={Index}

%FIXME
%\setupdelimitedtext[quote][2]
%  [left={\symbol[leftquotation]},
%   right={\symbol[rightquotation]}]
%\setupdelimitedtext[quotation][2]
%  [left={\symbol[leftquote]},
%   right={\symbol[rightquote]}]

% -----------------------------------------------------------------------------
% Typeface TODO
% -----------------------------------------------------------------------------

\define\setupfontfeature
  {\dodoubleargument\dosetupfontfeature}

\starttexdefinition dosetupfontfeature [#1][#2]
  \definefontfeature[#1][#1]
    [#2]
\stoptexdefinition

\definefontfeature[tabular][default]
  [tnum=yes]
\definefont[tb][Serif*tabular]

% -----------------------------------------------------------------------------
% Directories
% -----------------------------------------------------------------------------

\setupexternalfigures
  [directory={figuren,afbeeldingen,figures,images,
              ../figuren,../afbeeldingen,../figures,../images}]

% -----------------------------------------------------------------------------
% Emphasize, Alert & URL's
% -----------------------------------------------------------------------------

\definehighlight[emph]
  [style=\em]
\definehighlight[alert]
  [style=\bf]

\definestartstop[emphasize]
  [style=\em]

\define[1]\online
  {\useurl[dummy][#1]
   \goto{\url[dummy]}[url(#1)]}
\define[1]\mailto
  {\useurl[dummy][#1]
   \goto{\url[dummy]}[url(mailto:#1)]}

\defineparagraphs[sidebysidetext]
  [n=2]

\define[1]\introduce
  {\infull{#1} (\getvalue{#1})}

% -----------------------------------------------------------------------------
% Todo's
% -----------------------------------------------------------------------------

%\definelist[todo]
  %[alternative=a,
   %partnumber=no,
   %pagestyle=\quad\it]
%\definecombinedlist[todos]
  %[chapter,todo]
%\setupcombinedlist[todos]
  %[level=current,% Waarom weet ik ook niet, maar het werkt.
   %interaction=pagenumber]

\defineregister[todos]
\setupregister[todos]
  [n=1,
   indicator=no,
   before={\blank[none]}]

\starttexdefinition todo
  \dodoublegroupempty\dotodo
\stoptexdefinition

\starttexdefinition dotodo #1#2
  \ifsecondargument
    \inmargin{#1}
    \underbar{#2}
    \todos{#1+#2}
  \else
    \inmargin{Todo}
    \underbar{#1}
    \todos{Todo+#1}
  \fi
\stoptexdefinition

% -----------------------------------------------------------------------------
% Referencing
% -----------------------------------------------------------------------------

\starttexdefinition see [#1:#2]
  \doifdefinedelse{in#1}
    {\getvalue{in#1}[#1:#2]}
    {\writestatus{warning}{referenceformat in#1 not defined}
     \in[#1:#2]}
\stoptexdefinition % #3 is optional and gobbled bij \in

\starttexdefinition seealso [#1:#2]
  (\labeltext{seealso}\see[#1:#2])
\stoptexdefinition
%\define\seealso
  %{\groupedcommand{(\labeltext{seealso}\see}{)}}

\definereferenceformat[insec]
  [left=§]
\definereferenceformat[inchp]
  [text=\word{\labeltext{chapter}}]
\definereferenceformat[inapp]
  [text=\word{\labeltext{appendix}}]
\definereferenceformat[infig]
  [text=\word{\labeltext{figure}}]
\definereferenceformat[intab]
  [text=\word{\labeltext{table}}]
\definereferenceformat[infor]
  [left=(,
   right=)]

% -----------------------------------------------------------------------------
% Math Commands
% -----------------------------------------------------------------------------

\let\phi\varphi
\let\epsilon\varepsilon
\let\hbar\hslash
\let\BIG\bigg

\define\where
  {{\bf where}}

\define\booleans
  {{\math{\mathbb{B}}}}
\define\identifiers
  {{\math{\mathbb{I}}}}

\let\lbracket[
\let\rbracket]

\define\lAngle
  {\math{\langle\!\langle}}
\define\rAngle
  {\math{\rangle\!\rangle}}
\define\lBracket
  {\math{\lbracket\!\lbracket}}
\define\rBracket
  {\math{\rbracket\!\rbracket}}

\define\<{\math{\langle}}
\define\>{\math{\rangle}}
\define\[{\math{\lBracket}}
\define\]{\math{\rBracket}}

\define\*{\math{^\star}}
\define\+{\math{^+}}
\define\?{\math{^?}}
\define\|{\math{\mid}}

\definemathcommand[arcsinh][nolop]
  {\mfunction{arcsinh}}
\definemathcommand[arccosh][nolop]
  {\mfunction{arccosh}}
\definemathcommand[arctanh][nolop]
  {\mfunction{arctanh}}
\definemathcommand[e][nolop]
  {\mfunction{e}}

\definemathcommand[erf][nolop]
  {\mfunction{erf}}
\definemathcommand[erfi][nolop]
  {\mfunction{erfi}}
\definemathcommand[frec][nolop]
  {\mfunction{frec}}
\definemathcommand[fres][nolop]
  {\mfunction{fres}}

\define\total
  {{\rm d}}
 
\starttexdefinition diff
  \doquadruplegroupempty\dodiff
\stoptexdefinition
\starttexdefinition dodiff #1#2#3#4
  \iffourthargument
    \frac{#1^{#2} #3}{#1 #4^{#2}}
  \else
    \frac{#1 #2}{#1 #3}
  \fi
\stoptexdefinition

\define\tdiff
  {\diff{\total}}
\define\pdiff
  {\diff{\partial}}
\define\ddiff
  {\diff{\total}{}}

\redefine\d
  {\;\total}
\define\D
  {\;{\cal D}}
\define\p
  {\;\partial}

%FIXME: cannot be \redefine'ed
%\define[1]\vec
%  {{\bf #1}}
%\define[1]\mat
%  {\text{\ss\bf #1}}
%\define[1]\op
  %{\hat{#1}}

\define\cross
  {\times}
\define\Grad
  {\vec{\nabla}}
\define[1]\Div
  {\vec{\nabla}\cdot\vec{#1}}
\define[1]\Curl
  {\vec{\nabla}\cross\vec{#1}}
\define\Lapl
  {\vec{\nabla}^2}

\redefine\implies
  {\Rightarrow}
\redefine\iff
  {\Leftrightarrow}
%\define\to
%\define\mapsto
%\define\leadsto

\define[1]\E
  {\math{\cdot 10^{#1}}}
\define[1]\reci
  {\math{\frac{1}{#1}}}
\define[1]\unitfrac
  {{\textstyle\frac{1}{#1}}}
\define\half
  {\reci{2}}
\define\third
  {\reci{3}}
\define\rpi
  {\reci{\pi}}
\define\halfpi
  {\frac{\pi}{2}}
\define\rhbar
  {\reci{\hbar}}
\define\rhslash
  {\reci{\hslash}}

% -----------------------------------------------------------------------------
% Math Alignment
% -----------------------------------------------------------------------------

\definemathalignment[mathspreading]
  [n=1,
   align=left,
   distance=1em plus 1 fill]

\definemathmatrix[aligned]
  [distance=1pt,
   align=left,
   %location=top,
   style=\displaystyle]

% Formulas where right hand side should be splitted over two or more lines:
% a = b
%   = c
%   = d
% Read for \SC Split Column and for \SR Split Row.

\definecomplexorsimpleempty\startsplit

\starttexdefinition complexstartsplit [#1]
  \start
    \define\SC{\NC}
    \define\SR{\NR\NC\NC}
    \startmathalignment[#1]
      \NC
\stoptexdefinition

\starttexdefinition stopsplit
      \NR[+]
    \stopmathalignment
  \stop
\stoptexdefinition

%\starttexdefinition startsplittedformula
%  \startformula\startsplit
%\stoptexdefinition
%\starttexdefinition stopsplittedformula
%  \stopsplit\stopformula
%\stoptexdefinition

% Multiple formulas spreaded over the line:
% a = b    c = d    e = f
% Read for \SC Spread Column and for \SR Spread Row.

\definecomplexorsimpleempty\startspread

\starttexdefinition complexstartspread [#1]
  \start
    \define\SC{\NC}
    \define\SR{\NR\NC}
    \startmathspreading[m=2,#1]
      \NC
\stoptexdefinition

\starttexdefinition stopspread
      \NR[+]
    \stopmathspreading
  \stop
\stoptexdefinition

%\starttexdefinition startspreadedformula
%  \startformula\startspread
%\stoptexdefinition
%\starttexdefinition stopspreadedformula
%  \stopspread\stopformula
%\stoptexdefinition

% Formulas denoting step-by-step derivations.
%    a = b
% => b = c
% => c = d
% Read for \SC Step Column and \SR Step Row

\definecomplexorsimpleempty\startsteps

\starttexdefinition complexstartsteps [#1]
  \start
    \define\SC{\NC}
    \define\SR{\NR\NC\NC\implies\NC}
    \startmathalignment[m=2,#1]
      \NC\NC \NC
\stoptexdefinition

\starttexdefinition stopsteps
      \NR[+]
    \stopmathalignment
  \stop
\stoptexdefinition

% Formulas spread accross multiple lines
% a + b
%   + c + d
%   + e + f

\definecomplexorsimpleempty\startmultiline

\starttexdefinition complexstartmultiline [#1]
  \start
    \define\SR{\NR\NC\quad}
    \startmathalignment[n=1,align=left]
      \NC
\stoptexdefinition

\starttexdefinition stopmultiline
      \NR[+]
    \stopmathalignment
  \stop
\stoptexdefinition


% -----------------------------------------------------------------------------
% Math Extensions EXPERIMENTAL
% -----------------------------------------------------------------------------

% To keep spaces inside math formulas

\start
  \catcode\spaceasciicode\activecatcode
  \gdef\makespacespecial{\def {\ }}
\stop

\starttexdefinition keepspaces
  \catcode\spaceasciicode\othercatcode
  \makespacespecial
\stoptexdefinition

% -----------------------------------------------------------------------------
% Tabbed data
% -----------------------------------------------------------------------------

\usemodule[database]

\defineseparatedlist[tabbeddata]
  [separator=tab,
   first=\NC,
   right=\NC,
   last=\NR]

\starttexdefinition externaltabbeddata [#1]
  \processseparatedfile[tabbeddata][#1]
\stoptexdefinition

% -----------------------------------------------------------------------------
% TikZ pictures
% -----------------------------------------------------------------------------

\define\picture\tikz

\starttexdefinition startpicture
  \start
    \hbox\bgroup
      \dosingleempty\starttikzpicture
\stoptexdefinition

\starttexdefinition stoppicture
      \stoptikzpicture
    \egroup
  \stop
\stoptexdefinition

\starttexdefinition setuppicturestyle
  \dodoubleargument\dosetuppicturestyle
\stoptexdefinition

\starttexdefinition dosetuppicturestyle [#1][#2]
  \tikzset{#1/.style={#2}}
\stoptexdefinition

% -----------------------------------------------------------------------------
% Titles
% -----------------------------------------------------------------------------

\starttexdefinition completetitle
  \startstandardmakeup
    \titleelement{head}
    \vfill
    \placetitle
    \vfill
    \vfill
  \stopstandardmakeup
\stoptexdefinition

% -----------------------------------------------------------------------------
% Publications
% -----------------------------------------------------------------------------

\define\inlinecite
  {\cite[authoryear]}

\setuppublicationlayout[electronic]
  {\insertauthors{}{.\space}{}%
   \inserttitle{\bgroup\em}{\egroup.\space}{}%
   \inserturl{\bgroup\tt}{\egroup.\space}{}%
   \insertnote{\bgroup(}{)\egroup.}{}}

% -----------------------------------------------------------------------------
% Makeup
% -----------------------------------------------------------------------------

\starttexdefinition showalphabets
  \doalphabetvalues\rm
  \doalphabetvalues\ss
  \doalphabetvalues\tt
  \starttabulate[|l|rT|]
  \NC \getvalue{alphabet:text:rm} \NC \getvalue{alphabet:width:rm} \NR
  \NC \getvalue{alphabet:text:ss} \NC \getvalue{alphabet:width:ss} \NR
  \NC \getvalue{alphabet:text:tt} \NC \getvalue{alphabet:width:tt} \NR
  \stoptabulate
\stoptexdefinition

\starttexdefinition doalphabetvalues #1
  \setvalue{alphabet:text:\strippedcsname#1}
    {#1\dorecurse{26}{\character\recurselevel}}
  \setvalue{alphabet:width:\strippedcsname#1}
    {\setbox\scratchbox\hbox{\getvalue{alphabet:text:\strippedcsname#1}}
     \the\wd\scratchbox}
\stoptexdefinition

\stopenvironment

% vim: ft=context ts=2 sw=2 et
